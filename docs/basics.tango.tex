\subsection{Basic preamble}
Add some informations to your document.
\code{{{
  \title{Tango user's guide}
  \author{Nadarajah Mag-Stellon}
  \date{01 Avril 2014}
}}}

\subsection{Document}
As in Latex, you can use any class system for you document's command.
The basics document like article, report, book ... are available.
\code{{{
  \document{article}   
}}}

\subsection{Environment}
There are two kind of environment : anonymous and named environment
Anonymous environment is use with the keyword *abstract* and let you write a nameless environment.

\code{{{
  \begin{EnvironementA}
  \end{EnvironmentA}

  \begin{abstract}
  \end{abstract}
}}}

\subsection{Sections}
Tango provides commands which let you structure your document.
\code{{{
  \section{section A} or = section A =
  \subsection{subsection A} or == subSection A ==
  \part{part A}
  \chapter{chapter A}
  \subsubsection{subsection A}
  \paragraph{paragraph A}
}}}

\subsection{Styles}
You can use differents styles for your text.
\code{{{
  **double star bold text**
  __double underline bold text__

  *single star emph text*
  \emph{emph command text}
}}}

\subsection{List}
Two types of lists are available : itemize and enumerate.
In Markdown style :
\code{{{
  We see three *classical* functions : 
  - the factorial function 
  - the fibonacci function
  - the ackermann function

  1. the factorial highlights:
    a. a simple recursive scheme
    b. a tail-call variant with a single accumulator
    c. an imperative variant
      i) using a while loop
      ii) using a range iterator
  2. the fibonacci highlights:
    - a more complex recursive scheme
    - the use of two accumulators for the tail-call version 
}}}

In Latex style :
\code{{{
  \begin{itemize}
  \item the factorial function
  \item the fibonacci function
  \item the ackermann function
  \end{itemize}

  \begin{enumerate}
  \item the factorial hightlights
    \begin{enumerate}
    \item a simple recursive scheme
      ... etc.
    \end{enumerate}
  \end{enumerate}
}}}

\subsection{Comments}
Tango has single and multi line comments.
\code{{{
  % A single-line comment

  %{
  A multiline comment 
  %{ nested comment }%
  }%  
}}}

\subsection{Block code}
Show a code block and many languages are supported
\code{{{
def fact(n):
  if n = 0:
    return 1
  else:
    return n * fact(n-1)
}}}

A langage can be set by default
\code{{{
  \set[defaultcodelanguage=python]
}}}

You can create a reference to a block code.This reference
let you use it as a copy/paste.
Create a reference as below :
\code{{{
  \begin{code}[coderef=factit]  
    def fact(n):
    def factit(n,acc):
    if n = 0:
    return acc
    else:
    return factit(n-1,n*acc)
    # body of the function
    return factit(n,1)
  \end{code}
}}}
Use your reference as below :
\code{{{
\begin{evalcode}
  \coderef[factit]
  print("fact(4)={0}".format(fact(4))')
\end{evalcode}
}}}

\subsection{Maths}
Math expressions are rendered by MathJax.
Put your math expression into \$
\code{{{
   $n! = 1 \times \ldots \times n$
}}}

For bigger expression, follow the example below.
\code{{{
  \[
   \left \{ \begin[t]{array}{l}
              0!=1 \\
              n! = n \times (n-1)! \text{ for } n>0
            \end{array} \right.
  \]
}}}

\subsection{Commands}
Syntax for hyperlinks
\code{{{
  \url[Python]{http://www.python.org}
   [Python|http://www.python.org]
}}}
Syntax for footnote
\code{{{
  \footnote{footnote}
}}}
