\section{Basics}

\subsection{Basic preamble}
Add some informations to your document.
\begin{code}[langague=tango]
  \title{Tango user's guide}
  \author{Nadarajah Mag-Stellon}
  \date{01 Avril 2014}
\end{code}

\subsection{Document}
As in Latex, you can use any class system for you document's command.
The basics document like article, report, book ... are available.
\begin{code}[langague=tango]
  \document{article}   
\end{code}

\subsection{Environment}
There are two kind of environment : anonymous and named environment
Anonymous environment is use with the keyword *abstract* and let you write a nameless environment.
\begin{code}[langague=tango]
  \begin{EnvironementA}
  \end{EnvironmentA}

  \begin{abstract}
  \end{abstract}
\end{code}

\subsection{Sections}
Tango provides commands which let you structure your document.
\begin{code}[langague=tango]
  \section{section A} or = section A =
  \subsection{subsection A} or == subSection A ==
  \part{part A}
  \chapter{chapter A}
  \subsubsection{subsection A}
  \paragraph{paragraph A}
\end{code}

\subsection{Styles}
You can use differents styles for your text.
\begin{code}[langague=tango]
  **double star bold text**
  __double underline bold text__

  *single star emph text*
  \emph{emph command text}
\end{code}

\subsection{List}
Two types of lists are available : itemize and enumerate.

\begin{code}[langague=tango]
  \begin{itemize}
  \item the factorial function
  \item the fibonacci function
  \item the ackermann function
  \end{itemize}

  \begin{enumerate}
  \item the factorial hightlights
    \begin{enumerate}
    \item a simple recursive scheme
      ... etc.
    \end{enumerate}
  \end{enumerate}
\end{code}



We see three *classical* functions : %% *word* translates to \strong{word}  and then \textbf{word} by default
  - the factorial function %% at least two spaces leading the '-' symbol
  - the fibonacci function
  - the ackermann function
%{ this translates to
\begin{itemize}
\item the factorial function
\item the fibonacci function
\item the ackermann function
\end{itemize}
%}

This allows to show the following features.
  1. the factorial highlights:
    a. a simple recursive scheme  % at least 2 chars more
    b. a tail-call variant with a single accumulator
    c. an imperative variant
      i) using a while loop  % this is the final nesting, one must use the latex syntax for deeper nesting
      ii) using a range iterator
  2. the fibonacci highlights:
    - a more complex recursive scheme
    - the use of two accumulators for the tail-call version %  an embedded itemize
%{ this translates to
\begin{enumerate}
\item the factorial hightlights
\begin{enumerate}
\item a simple recursive scheme

... etc.

\end{enumerate}
\end{enumerate}
%}




\subsection{Commands}

\begin{code}[langague=tango]
\url[Python]{http://www.python.org}
\end{code}

\subsection{Comments}

% A single-line comment starts with % and extends until the end of the line

%{
A multiline comment start with %{  and ends  with the next }%.
And multiline comments can be nested of course.
}%

 [Python|http://www.python.org]
% a wiki-like syntax for hyperlinks is supported


\begin{code}[language=python] % a code block, support for various programming languages is planned
def fact(n):
  if n = 0:
    return 1
  else:
    return n * fact(n-1)
\end{code}


% the langage by default can be set
\set[defaultcodelanguage=python]



An tail-call version is as follows\footnote{The python interpreter does not eliminate tail calls, but the notion remains interesting anyway.}:


% default language is python now
\begin{code}[coderef=factit]  
def fact(n):
  def factit(n,acc):
    if n = 0:
      return acc
   else:
      return factit(n-1,n*acc)
  # body of the function
  return factit(n,1)
\end{code}

\begin{evalcode} % a block of code can be evaluated and its (by default, standard) output printed to ScriptTex
# insertion of code blocks
\coderef[factit]
print("fact(4)={0}".format(fact(4))')
\end{evalcode} % the result of an evalcode must be executable in isolation

\section{Basics}
