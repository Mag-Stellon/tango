
\chapter[label=chap:architecture]{Architecture du logiciel}

Le logiciel Tango prend en entrée un document décrit dans le langage _TangoTex_  inspiré de Latex et enrichi par des idées provenant du format _Markdown_.

Le document source en entrée est «avalé» par Tango et traité de la façon suivante:

  1. le _parseur_ (analyseur lexical et syntaxique) de Tango transforme le texte du document en une représentation interne structurée (arbre de syntaxe abstraite ou AST)

  2. le _processeur_ parcourt la représentation structurée et la transforme pour réaliser des modifications du document.

  3. le _générateur_ récupère le document après transformation par le processeur, et génère un document cible permettant d'être affiché sur écran ou imprimé.


Les deux étapes du _processeur_ et du _générateur_ sont hautement paramétrables.

