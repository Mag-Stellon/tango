
%% This is a simple example of various ScriptTex features

% A single-line comment starts with % and extends until the end of the line

% ScriptTex "understands" a language largely  inspired by tex/latex
% however only a small subset of the tex/latex functionalities
% are supported

%{
A multiline comment start with %{  and ends  with the next }%.
And multiline comments can be nested of course.
}%

\document{article}   % also  report, book  and maybe others later (class system ?)

\title{Simple recursive functions in Python\\
An example of some ScripTex features}

\author{Frédéric Peschanski}  %% remark : ScriptTex only supports UTF-8 encodings

\date{\today}


= Introduction =
%{ the same as :
\section{Basic functions}
in article mode
}%

We describe a few simple _recursive_ functions in the [Python|http://www.python.org] language. % _word_ translates to \emph{word}
% a wiki-like syntax for hyperlinks is supported

We see three *classical* functions : %% *word* translates to \strong{word}  and then \textbf{word} by default
  - the factorial function %% at least two spaces leading the '-' symbol
  - the fibonacci function
  - the ackermann function
%{ this translates to
\begin{itemize}
\item the factorial function
\item the fibonacci function
\item the ackermann function
\end{itemize}
%}

This allows to show the following features.
  1. the factorial highlights:
    a. a simple recursive scheme  % at least 2 chars more
    b. a tail-call variant with a single accumulator
    c. an imperative variant
      i) using a while loop  % this is the final nesting, one must use the latex syntax for deeper nesting
      ii) using a range iterator
  2. the fibonacci highlights:
    - a more complex recursive scheme
    - the use of two accumulators for the tail-call version %  an embedded itemize
%{ this translates to
\begin{enumerate}
\item the factorial hightlights
\begin{enumerate}
\item a simple recursive scheme

... etc.

\end{enumerate}
\end{enumerate}
%}

== Factorial ==
%{
The same as :
\subsection{Factorial}
in article mode
}%
The factorial function is often denoted $n! = 1 \times \ldots \times n$
% ScriptTex is interfaced with MathJax for HTML/CSS/JS output of embedded maths
% it relies on pdflatex or luatex for the pdf output

A more constructive version is:

\[
\left \{ \begin[t]{array}{l}
0!=1 \\
n! = n \times (n-1)! \text{ for } n>0
\end{array} \right.
\] % both display math and inline math are supported

This can be almost directly translated to python code :

\begin{code}[language=python] % a code block, support for various programming languages is planned
def fact(n):
  if n = 0:
    return 1
  else:
    return n * fact(n-1)
\end{code}

The name of the function is `fact` and its definition is quite simple. % the `word` sequence is translated to 
%  \verbUwordU in Latex mode, where the U symbol is choosen fresh.  Not that line feeds are disallowed.

% the langage by default can be set
\set[defaultcodelanguage=python]

An tail-call version is as follows\footnote{The python interpreter does not eliminate tail calls, but the notion remains interesting anyway.}:

% default language is python now
\begin{code}[coderef=factit]  
def fact(n):
  def factit(n,acc):
    if n = 0:
      return acc
   else:
      return factit(n-1,n*acc)
  # body of the function
  return factit(n,1)
\end{code}

\begin{evalcode} % a block of code can be evaluated and its (by default, standard) output printed to ScriptTex
# insertion of code blocks
\coderef[factit]
print("fact(4)={0}".format(fact(4))')
\end{evalcode} % the result of an evalcode must be executable in isolation

