
\begin{exercise}[title=Listes d'intervalles]

\begin{question}
Définir une fonction \snip{construire} qui, étant donné un entier naturel \snip{n} strictement positif, retourne la liste
des entiers de \snip{1} à \snip{n} (inclus).

\defPython[construire]{
def construire(n):
    """Nat -> LISTE[Nat]
        Hypothèse: n > 0
        retourne la liste des entiers naturels 1 à n (inclus)
    """
    l = []
    for i in range(1,n+1):
         l.append(i)
    return l
}

Par exemple :

\checkPython{
>>> construire(5)
[1,2,3,4,5]
>>> construire(1)
[1]
}


\answer[9]{
\showDefPython[construire]

Une autre solution sans utiliser \snip{range} :

\showEvalPython{
def construire(n):
    l = []
    i = 1
    while i <= n:
        l.append(i)
        i += 1
    return l

construire(5)
construire(1)
}

}

\end{question}

\begin{question}
Definir une fonction 
\end{question}

\end{exercise}