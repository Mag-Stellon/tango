Tango user's guide
======


------


Synopsis
------

python3 [documentName] [tango.py path] [--html | --tex | --xml]


Description
------

Tango is a programmable document processor which provides Latex/HTML output.
Tango use documents written in a mix of latex and markdown-inspired short cuts.


Using Pandoc
------

Before using Tango, you have to be into your document direction.
After, open a terminal and launch the **Synopsis** command-line.

Options
------

#### --html
Produce a HTML output

####
Produce a Latex output

#### --xml
Produce a xml output which is useful for debugging or pretty-print.




Tango syntax
------

If you get to know Markdown and Latex, it's quite easy since tango is inspired by them.

#### Headers


\section{section A}
Section A text ...

= Section B =
Section B text ...

\subsection{subsection A}
Subsection A text ...

\part{part A}
Part A text ...

\chapter{chapter A}
Chapter A text ...

\subsubsection{subsection A}
Subsubsection A text

\paragraph{paragraph A}
Paragrah A text

\subparagraph{paragraph A}
Subparagrah A text

#### Styles

**double star bold text**
__double underline bold text__

*single star emph text*
\emph{emph command text}


#### Lists

List A :
\begin{itemize}
\item the factorial function
\item the fibonacci function
\item the ackermann function
\end{itemize}

Enumeration A :
\begin{enumerate}
\item the factorial hightlights
\begin{enumerate}
\item a simple recursive scheme
... etc.
\end{enumerate}
\end{enumerate}

#### Extension


###### URL
\url[Python]{http://www.python.org}


